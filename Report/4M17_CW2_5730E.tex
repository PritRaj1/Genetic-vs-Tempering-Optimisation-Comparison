\documentclass[10pt]{article}
\usepackage{titling} % Customize the title
\usepackage[utf8]{inputenc}
\usepackage{eso-pic}
\usepackage{charter}
\usepackage[margin=1in]{geometry}
\usepackage{amssymb,pdfpages,fancyhdr,subcaption,graphicx,hyperref,float,outlines,amsmath,gensymb}
\usepackage{listings}
\usepackage{parskip}
\usepackage{multicol} % Added the multicol package
\usepackage{booktabs}
\usepackage{graphicx}
\usepackage{subcaption}
\usepackage{multirow}
\usepackage{listings}
\usepackage{xcolor}

% Define colors for code highlighting
\definecolor{codegreen}{rgb}{0,0.6,0}
\definecolor{codegray}{rgb}{0.5,0.5,0.5}
\definecolor{codepurple}{rgb}{0.58,0,0.82}
\definecolor{backcolour}{rgb}{0.95,0.95,0.92}

% Define settings for Python code
\lstdefinestyle{mystyle}{
    backgroundcolor=\color{backcolour},
    commentstyle=\color{codegreen},
    keywordstyle=\color{blue},
    numberstyle=\tiny\color{codegray},
    stringstyle=\color{codepurple},
    basicstyle=\ttfamily\footnotesize,
    breakatwhitespace=false,
    breaklines=true,
    captionpos=b,
    keepspaces=true,
    numbers=left,
    numbersep=5pt,
    showspaces=false,
    showstringspaces=false,
    showtabs=false,
    tabsize=2
}

\lstset{style=mystyle}

\title{\vspace{-3cm}\textbf{4M17 Coursework \#2 - Optimisation Algorithm Performance Comparison}}
\author{\vspace{-3cm}\textbf{Candidate No: 5730E}}

\begin{document}
\maketitle
\section{Abstract}
This report conducts a comparative analysis of two optimisation algorithms applied to minimise Keane's Bump Function, (KBF). In particular, the study focuses on a Continuous Genetic Algorithm, (GA), as well as an alternative algorithm not covered in the lectures: the State Transition Algorithm, (STA).

\begin{multicols}{2}

\section{Introduction}
\subsection{Keane's Bump Function}
\begin{figure}[H]
    \centering
    \begin{subfigure}{0.49\textwidth}
        \centering
        \includegraphics[width=\textwidth]{../figures/KBF_function_surf.png}
        \caption{Surface plot.}
        \label{fig:KBF_surf}
    \end{subfigure}
    \begin{subfigure}{0.49\textwidth}
        \centering
        \includegraphics[width=\textwidth]{../figures/KBF_function_contour.png}
        \caption{Contour plot.}
        \label{fig:KBF_contour}
    \end{subfigure}
    \captionsetup{justification=centering}
    \caption{The two-dimensional visualisation of the Keane's Bump Function, (KBF).}
    \label{fig:KBF_2D}
\end{figure}
To compare the performances of the two algorithms, the Keane's Bump Function, (KBF), is used as the objective function. In particular, the n-dimensional constrained optimisation problem is defined as the maximisation of:

\begin{equation}
    f(\mathbf{x}) = \left| \frac{\sum_{i=1}^{n} (\cos(x_i))^4 - 2\prod_{i=1}^{n} (\cos(x_i))^2}{\sqrt{\sum_{i=1}^{n} i \cdot x_i^2}} \right| \\\vspace{1cm}
    \label{eq:KBF_cost}
\end{equation}
\begin{equation}
    \begin{aligned}
        \text{subject to} \quad & 0 \leq x_i \leq 10 \quad \forall i \in \{1, \dots, n\} \\
        & \quad \prod_{i=1}^{n} x_i > 0.75 \\
        & \quad \sum_{i=1}^{n} x_i < \frac{15n}{2}
    \end{aligned} 
    \label{eq:KBF_constraints}
\end{equation}

The two-dimensional form of the function has been plotted in Figure \ref{fig:KBF_2D}. Some notable properties are as follows:

\begin{itemize}
    \item The function is undefined at the origin, (0, 0). This is due to the division by zero in the denominator of Equation \ref{eq:KBF_cost}. Otherwise, the function is continuous and differentiable everywhere.
    \item The function is highly multi-modal. Its global maximum is located at the constaint boundary $x_{n}=0$, where $x_n$ denotes the final variable in the n-dimensional space. However, there are many local maxima located inside the feasible region, all of which have quite similar amplitudes.
    \item The function is nearly symmetric about the line $x_1=x_2$. This stems from its construction in \ref{eq:KBF_cost}, which primarily involves the the squares of individual input variables, $x_i^2$. This results some invariance reagrding the order of the input variables. Overall, the peaks consistently manifest in pairs, yet there is a notable pattern wherein one peak always surpasses its counterpart in magnitude.
\end{itemize}

Given the above properties, the KBF is a challenging function to optimise. The presence of multiple, similar-amplitude local maxima makes it difficult for the optimisation algorithm to converge to the global maximum. On the other hand, all control variables share the same nature, (continuous variables), and exhibit identical scales. Additionally, all constraints are of the inequality type, and the feasible space is non-disjoint.

These properties make the KBF a suitable candidate for the comparative analysis of the two optimisation algorithms, as discussed in the previous work \cite{ELBELTAGY1999639}.

\subsection{Continuous Genetic Algorithm}

\end{multicols}

\section{Methodology}
\section{Results}
\section{Discussion}
\section{Conclusion}
\bibliographystyle{plain} % We choose the "plain" reference style
\bibliography{refs} % Entries are in the refs.bib file
\end{document}